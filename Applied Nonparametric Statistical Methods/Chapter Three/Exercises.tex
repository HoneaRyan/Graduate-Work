\documentclass[]{article}
\usepackage{lmodern}
\usepackage{amssymb,amsmath}
\usepackage{ifxetex,ifluatex}
\usepackage{fixltx2e} % provides \textsubscript
\ifnum 0\ifxetex 1\fi\ifluatex 1\fi=0 % if pdftex
  \usepackage[T1]{fontenc}
  \usepackage[utf8]{inputenc}
\else % if luatex or xelatex
  \ifxetex
    \usepackage{mathspec}
  \else
    \usepackage{fontspec}
  \fi
  \defaultfontfeatures{Ligatures=TeX,Scale=MatchLowercase}
\fi
% use upquote if available, for straight quotes in verbatim environments
\IfFileExists{upquote.sty}{\usepackage{upquote}}{}
% use microtype if available
\IfFileExists{microtype.sty}{%
\usepackage{microtype}
\UseMicrotypeSet[protrusion]{basicmath} % disable protrusion for tt fonts
}{}
\usepackage[margin=1in]{geometry}
\usepackage{hyperref}
\hypersetup{unicode=true,
            pdftitle={Chapter Three},
            pdfauthor={Ryan B. Honea},
            pdfborder={0 0 0},
            breaklinks=true}
\urlstyle{same}  % don't use monospace font for urls
\usepackage{graphicx,grffile}
\makeatletter
\def\maxwidth{\ifdim\Gin@nat@width>\linewidth\linewidth\else\Gin@nat@width\fi}
\def\maxheight{\ifdim\Gin@nat@height>\textheight\textheight\else\Gin@nat@height\fi}
\makeatother
% Scale images if necessary, so that they will not overflow the page
% margins by default, and it is still possible to overwrite the defaults
% using explicit options in \includegraphics[width, height, ...]{}
\setkeys{Gin}{width=\maxwidth,height=\maxheight,keepaspectratio}
\IfFileExists{parskip.sty}{%
\usepackage{parskip}
}{% else
\setlength{\parindent}{0pt}
\setlength{\parskip}{6pt plus 2pt minus 1pt}
}
\setlength{\emergencystretch}{3em}  % prevent overfull lines
\providecommand{\tightlist}{%
  \setlength{\itemsep}{0pt}\setlength{\parskip}{0pt}}
\setcounter{secnumdepth}{0}
% Redefines (sub)paragraphs to behave more like sections
\ifx\paragraph\undefined\else
\let\oldparagraph\paragraph
\renewcommand{\paragraph}[1]{\oldparagraph{#1}\mbox{}}
\fi
\ifx\subparagraph\undefined\else
\let\oldsubparagraph\subparagraph
\renewcommand{\subparagraph}[1]{\oldsubparagraph{#1}\mbox{}}
\fi

%%% Use protect on footnotes to avoid problems with footnotes in titles
\let\rmarkdownfootnote\footnote%
\def\footnote{\protect\rmarkdownfootnote}

%%% Change title format to be more compact
\usepackage{titling}

% Create subtitle command for use in maketitle
\newcommand{\subtitle}[1]{
  \posttitle{
    \begin{center}\large#1\end{center}
    }
}

\setlength{\droptitle}{-2em}
  \title{Chapter Three}
  \pretitle{\vspace{\droptitle}\centering\huge}
  \posttitle{\par}
  \author{Ryan B. Honea}
  \preauthor{\centering\large\emph}
  \postauthor{\par}
  \date{}
  \predate{}\postdate{}


\begin{document}
\maketitle

\subsection{Exercise One}\label{exercise-one}

\paragraph{Question}\label{question}

Verify the expressions given for the mean and variance in (3.2) for
statistics based on sums \(S_+\) or \(S_-\) of signed scores
\(s_i, 1 = 1,2,...,n\) where the sign associated with each \(s_i\) is
equally likely to be plus or minus.

\paragraph{Solution}\label{solution}

\subsection{Exercise Two}\label{exercise-two}

\paragraph{Question}\label{question-1}

In Comment 4 on Example 3.10 we asserted that the usual \(t\)-statistic
could be used in place of \(S_+\) as the test statistics for the Pitman
test because there was a one-to-one correspondence between the ordering
of the two statistics. Establish that this is so. (Hint: show that the
denominator of the \(t\)-statistic is invariant under all permutations
of the signs of the derivations \(d_i\).)

\paragraph{Solution}\label{solution-1}

\subsection{Exercise Three}\label{exercise-three}

\paragraph{Question}\label{question-2}

In Comment 3 on Example 3.2 we suggested that for a variety of reasons
one should be cautious about extending inferences about heartbeat rates
for female students to the population at large. What might some of these
reasons be?

\paragraph{Solution}\label{solution-2}

\subsection{Exercise Four}\label{exercise-four}

\paragraph{Question}\label{question-3}

Using the data in in Table 3.1 for the distribution of the Wilcoxon
\(S\) when \(n = 7\), construct a bar chart like that in Figure 3.1
showing the probability function for \(S\). Discuss the similarity, or
lack of similarity, to a normal distribution probability density
function.

\paragraph{Solution}\label{solution-3}

\subsection{Exercise Five}\label{exercise-five}

\paragraph{Question}\label{question-4}

Establish that the permutation distribution of the Wilcoxon signed-rank
statistic for testing the hypothesis \(H_0: \theta = 6\), given the
observations \[4\quad4\quad8\quad8\quad8\quad8\quad8\] has a
distribution equivelant to that fo rthe sign test of the same
hypothesis. Would this equivelance hold if the null hypothesis was
changed to \(H_0: \theta = 7\)?

\paragraph{Solution}\label{solution-4}

\subsection{Exercise Six}\label{exercise-six}

\paragraph{Question}\label{question-5}

Establish nominal 95 percent confidence intervals for the median based
on the Wilcoxon signed-rank test for the following data sets. If an
appropriate computer program is available use it to comment on the
discontinuities at the end points of your estimated intervals based on
Walsh averages.
\[\text{Set I }\quad1\quad1\quad1\quad1\quad1\quad3\quad3\quad5\quad5\quad7\quad7 \]
\[\text{Set II}\quad1\quad2\quad2\quad4\quad4\quad4\quad4\quad5\quad5\quad5\quad7 \]

\paragraph{Solution}\label{solution-5}

\subsection{Exercise Seven}\label{exercise-seven}

\paragraph{Question}\label{question-6}

Form a table of Walsh Averages for the Fisher sentence length data given
in Example 3.6, and use it to obtain 95 and 99 percent cofidence
intervals.

\paragraph{Solution}\label{solution-6}

\subsection{Exercise Eight}\label{exercise-eight}

\paragraph{Question}\label{question-7}

The numbers of pages in the sample of 12 books given in Exercise 2.5
were
\[126\quad142\quad156\quad228\quad245\quad246\quad370\quad419\quad433\quad454\quad478\quad503\]
Use the Wilcoxon signed-rank test to test the hypothesis that the mean
number of pages in the statistics books in the library from which the
sample was taken is 400. Obtain a 95 percent confidence interval for the
mean number of pages based on this Wilcoxon test and compare it with the
interval obtained using a \(t\)-test under an assumption of normality.

\paragraph{Solution}\label{solution-7}

\subsection{Exercise Nine}\label{exercise-nine}

\paragraph{Question}\label{question-8}

Apply the sign test to the data in Example 3.2 for the hypotheses
considered there.

\subsection{Exercise Ten}\label{exercise-ten}

\paragraph{Question}\label{question-9}

For the sample of 20 in Example 3.7 if \(\theta\) is the population
median test the hypothesis \(H_0: \theta = 9\) against the alternative
\(H_1: \theta \neq 9\) using the sign test by computing any relevant
binomial pobabilities directly from the binomial probability formula. It
is not necessary to determine the complete distribution to obtain the
relevant \(P\) value. To perform the test for \(H_0: \theta = 7.5\)
against the alternative \(H_1: \theta > 7.5\) additional terms in the
distribution will be needed. Either calculate these, or use tables or
computer software to carry out the appropriate test.

\subsection{Exercise Eleven}\label{exercise-eleven}

\paragraph{Question}\label{question-10}

Before treatment with a new drug 11 people with sleep problems have a
median sleeping time of 2 hours per night. A drug is administered and it
is known for good scientific reasons that it ahs an effct it will
increase sleeping time but some doctors doubt it will have any effect.
Are their doubts justified if the hours per night slept by these
individuals after taking the drug are: \[
3.1\quad1.8\quad2.7\quad2.4\quad2.9\quad0.2\quad3.7\quad5.1\quad8.3\quad2.1\quad2.4
\]

\subsection{Exercise Twelve}\label{exercise-twelve}

\paragraph{Question}\label{question-11}

Kimura and Chikuni (1987) give data for lengths of Greenland turbot of
various ages sampled from commercial catches in the Bering Sea as aaged
and measured by the Northwest and Alaska Fisheries Center. for
12-year-old turbot the numbers of each length were:

\[\begin{array}
{cccccccccc}
\text{Length (cm)} & 64 & 65 & 66 & 67 & 68 & 69 & 70 & 71 & 72 & 73 & 75 & 77 & 78 & 83 \\
\text{No. of Fish} & 1 & 2 & 1 & 1 & 4 & 3 & 4 & 5 & 3 & 3 & 1 & 6 & 1 & 1 \\
\end{array}
\]

Would you agree with someone who asserted that, on this evidence, the
median length of 12-year-old Greenland Turbot was almost certanly
between 69 and 72 cm?

\subsection{Exercise Thirteen}\label{exercise-thirteen}

\paragraph{Question}\label{question-12}

Use the Wilcoxon signed-rank test to test the hypothesis that the median
length of 12-year-old turbots is 73.5 using the data in Exercise 12.

\subsection{Exercise Fourteen}\label{exercise-fourteen}

\paragraph{Question}\label{question-13}

The first application listed in Section 3.7 involved British insurance
claims. The 2005 median was £1570. A random sample of 14 claims from a
large batch received in the first quarter of 2006 were for the following
amounts (in £): \[
1175\quad1183\quad1327\quad1581\quad1592\quad1624\quad1777\quad1924\quad2483\quad2642\quad2713\quad3419\quad5250\quad7615
\] What test do you consider appropriate fo ra shift in median relative
to the 2005 median? Would a one-tail test be appropriate? Obtain a 95
percent confidence interval for the median based upon these data. If all
amounts were converted to, say, Euros or to \$US, would your conclusions
be the same?

\subsection{Exercise Fifteen}\label{exercise-fifteen}

\paragraph{Question}\label{question-14}

The weight losses in kilograms for 16 overweight women who have been on
a diet for 2 months are as follows: \[
4\quad6\quad3\quad1\quad2\quad5\quad4\quad0\quad3\quad6\quad3\quad1\quad7\quad2\quad5\quad6
\] The firm sponsoring the diet advertises ``Lose 5kg in 2 months''. In
a consumer affairs radio programme, the claim this is an ``average''
weight loss. You may be unclear as to what the sponsors mean by
``average'', but assuming the sample size is effectively random do the
data support a median weight loss of 5 kg in the population of dieters?
Test this without an assumption of symmetry. What would be a more
appropriate test with an assumption of symmetry? Carry out this latter
test.

\subsection{Exercise Sixteen}\label{exercise-sixteen}

\paragraph{Question}\label{question-15}

A pathologist counts the numbers of diseased plants in randomly selected
areas each 1 metre square on a large field. For 40 such areas the
numbers of diseased plants are:

\[\begin{array}
{cccccccccc}
21 & 18 & 42 & 29 & 81 & 12 & 94 & 117 & 88 & 210 \\
44 & 39 & 11 & 83 & 42 & 93 & 2 & 11 & 33 & 91 \\
141 & 48 & 12 & 50 & 61 & 35 & 111 & 73 & 5 & 44\\
6 & 11 & 35 & 91 & 147 & 83 & 91 & 48 & 22 & 17
\end{array}
\]

Use histograms and boxplots to decide whether there is evidence of
skewness or outliers. Use nonparametric tests to find whether it is
reasonable to assume the median number of diseased plants per square
metre might be 50 (i) without assuming population symmetry, (ii)
assuming population symmetry. For these data and the evidence provided
by a boxplot and histograms do you consider the latter assumption
reasonable?

\subsection{Exercise Seventeen}\label{exercise-seventeen}

\paragraph{Question}\label{question-16}

A parking attendant notes the time cars have been illegaly parked after
their metered time has expired. For 16 offending cars, he records the
time in minutes as:
\[ 10\quad42\quad29\quad11\quad63\quad145\quad11\quad8\quad23\quad17\quad5\quad20\quad15\quad36\quad32\quad15 \]
Obtain an appropriate 95 percent confidence interval for the median
overstay time of offenders prior to detection. What assumptions were you
making to justify using the method you did? To what populaton do you
think the confidence interval you obtained might apply?

\subsection{Exercise Eighteen}\label{exercise-eighteen}

\paragraph{Question}\label{question-17}

Knapp (1982) gives the eprcentage of births on each day of the year
averaged over 28 years for Monroe County, new York. Ignoring leap years
(which makes little difference), the median percentage of births per day
is 0.2746. Not suprisingly, this is close to the expected percentage on
the assumption that births are equally like to be on any day, that is,
\(100/365 \approx 0.274\). We give below the average percentage for each
day in the month of September. If births are equally like to be on any
day of the year, this should resemble a random sample from a population
with median 0.2746. Do these data confirm this?

\[\begin{array}
{cccccccccc}
0.277 & 0.277 & 0.295 & 0.286 & 0.271 & 0.265 & 0.274 & 0.274 & 0.278 & 0.290\\
0.295 & 0.276 & 0.273 & 0.289 & 0.308 & 0.301 & 0.302 & 0.293 & 0.296 & 0.288\\
0.305 & 0.283 & 0.309 & 0.299 & 0.287 & 0.309 & 0.294 & 0.288 & 0.298 & 0.289
\end{array}
\]

\subsection{Exercise Nineteen}\label{exercise-nineteen}

\paragraph{Question}\label{question-18}

A standard method for assembling a routine component in a manufacturing
process is known to take a trained operative a median time of 14.2
mintues to complete. A new method of assembling that same component is
proposed. Twenty operatives are given practice sessions using the new
method, after which each claims he or she fully understands the process.
They are then each timed for assembling one new component. The times in
minutes they take are: \[\begin{array}
{cccccccccc}
14.2 & 11.3 & 12.7 & 19.2 & 13.5 & 14.4 & 11.8 & 15.1 & 12.3 & 11.7\\
13.2 & 13.4 & 14.0 & 14.1 & 13.7 & 11.9 & 11.8 & 10.7 & 11.3 & 12.2
\end{array}
\] Form a boxplot for the data. Follow this by any tests you consider
appropriate to indicate whether the median assembly time in mass
production is likely to differ from 14.2. Also obtain nominal 95 percent
confidence intervals for the median time associated with any test you
use. Compare in each case the intervals based on exact procedures with
the related asymptotic approximate intervals.

\subsection{Exercise Twenty}\label{exercise-twenty}

\paragraph{Question}\label{question-19}

In a class of 52 students 36 are categorized as being voerweight. In an
attempt to reduce obesity these 36 all volunteer to go on a diet. Each
is weighed at the start of the diet period, and after 3 weeks any gain
or loss in weight is recorded. A negative sign indicates a loss. Form a
boxplot of the gain or loss in weight and comment on any features that
may influence a choice of a test to determine whether or not the diet
has induced a change in median weight. Use what you consider the most
appropriate method to test whether the diet has affected median weight.
The weight change (in kg) are: \[\begin{array}
{cccccccccc}
-1.2 & 1.4 & 0.2 & -0.7 & -6.4 & -2.7 & -8.6 & -1.7 & -2.2\\
0.1 & -0.4 & -4.2 & -1.6 & 1.2 & -1.3 & -2.4 & 3.1 & -0.2\\
-4.5 & -6.3 & -1.7 & 0.0 & 0.2 & -3.7 & 1.1 & -2.3 & -0.1\\
-7.3 & 0.2 & -1.4 & -0.9 & -2.0 & 0.0 & 1.1 & -0.3 & -1.1
\end{array}
\]

Obtain an appropriate nominal 95 percent confidence interval for the
median change in weight.

\subsection{Exercise Twenty One}\label{exercise-twenty-one}

\paragraph{Question}\label{question-20}

For the data on sentence lengths in Example 3.6 use the modified van der
Waerden scores procedure proposed in Section 3.5.2 to test the
hypothesis \(H_0: \theta = 30\) against \(H_1: \theta > 30\)


\end{document}
