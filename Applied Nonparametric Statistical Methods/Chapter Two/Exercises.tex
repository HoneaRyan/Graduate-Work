\documentclass[]{article}
\usepackage{lmodern}
\usepackage{amssymb,amsmath}
\usepackage{ifxetex,ifluatex}
\usepackage{fixltx2e} % provides \textsubscript
\ifnum 0\ifxetex 1\fi\ifluatex 1\fi=0 % if pdftex
  \usepackage[T1]{fontenc}
  \usepackage[utf8]{inputenc}
\else % if luatex or xelatex
  \ifxetex
    \usepackage{mathspec}
  \else
    \usepackage{fontspec}
  \fi
  \defaultfontfeatures{Ligatures=TeX,Scale=MatchLowercase}
\fi
% use upquote if available, for straight quotes in verbatim environments
\IfFileExists{upquote.sty}{\usepackage{upquote}}{}
% use microtype if available
\IfFileExists{microtype.sty}{%
\usepackage{microtype}
\UseMicrotypeSet[protrusion]{basicmath} % disable protrusion for tt fonts
}{}
\usepackage[margin=1in]{geometry}
\usepackage{hyperref}
\hypersetup{unicode=true,
            pdftitle={Chapter Two},
            pdfauthor={Ryan B. Honea},
            pdfborder={0 0 0},
            breaklinks=true}
\urlstyle{same}  % don't use monospace font for urls
\usepackage{graphicx,grffile}
\makeatletter
\def\maxwidth{\ifdim\Gin@nat@width>\linewidth\linewidth\else\Gin@nat@width\fi}
\def\maxheight{\ifdim\Gin@nat@height>\textheight\textheight\else\Gin@nat@height\fi}
\makeatother
% Scale images if necessary, so that they will not overflow the page
% margins by default, and it is still possible to overwrite the defaults
% using explicit options in \includegraphics[width, height, ...]{}
\setkeys{Gin}{width=\maxwidth,height=\maxheight,keepaspectratio}
\IfFileExists{parskip.sty}{%
\usepackage{parskip}
}{% else
\setlength{\parindent}{0pt}
\setlength{\parskip}{6pt plus 2pt minus 1pt}
}
\setlength{\emergencystretch}{3em}  % prevent overfull lines
\providecommand{\tightlist}{%
  \setlength{\itemsep}{0pt}\setlength{\parskip}{0pt}}
\setcounter{secnumdepth}{0}
% Redefines (sub)paragraphs to behave more like sections
\ifx\paragraph\undefined\else
\let\oldparagraph\paragraph
\renewcommand{\paragraph}[1]{\oldparagraph{#1}\mbox{}}
\fi
\ifx\subparagraph\undefined\else
\let\oldsubparagraph\subparagraph
\renewcommand{\subparagraph}[1]{\oldsubparagraph{#1}\mbox{}}
\fi

%%% Use protect on footnotes to avoid problems with footnotes in titles
\let\rmarkdownfootnote\footnote%
\def\footnote{\protect\rmarkdownfootnote}

%%% Change title format to be more compact
\usepackage{titling}

% Create subtitle command for use in maketitle
\newcommand{\subtitle}[1]{
  \posttitle{
    \begin{center}\large#1\end{center}
    }
}

\setlength{\droptitle}{-2em}
  \title{Chapter Two}
  \pretitle{\vspace{\droptitle}\centering\huge}
  \posttitle{\par}
  \author{Ryan B. Honea}
  \preauthor{\centering\large\emph}
  \postauthor{\par}
  \date{}
  \predate{}\postdate{}


\begin{document}
\maketitle

\subsection{Exercise One}\label{exercise-one}

\paragraph{Question}\label{question}

A new type of intensive physiotherapy is developed for individuals who
have undergone spinal surgery. Due to limited hospital resources it can
only be given to 3 out of 10 patients. The patients are aged:

\[
  15\quad21\quad26\quad32\quad39\quad45\quad52\quad60\quad70\quad82
\]

Explain how a permutation test could be used to investigate whether use
of physiotherapy is related to patient age, (i.e.) whether there is a
policy to give treatment to younger as opposed to older groups or
\emph{vice versa}). If the patients aged 15, 26, and 32 have the
intensive physiotherapy find the \(P\)-value for a two-tailed test of an
appropriate null hypothesis. Comment on your findings.

\paragraph{Solution}\label{solution}

\subsection{Exercise Two}\label{exercise-two}

\paragraph{Question}\label{question-1}

Suppose that the new drug under test in Example 2.1 has all the
ingredients of a standard drug at present in use and an additional
ingredient that has proved to be of use for a related disease, so that
it is reasonable to assume that the new drug will do at least as well as
the standard one, but may do better. Formulate the hypothesses leading
to an appropriate one-tail test. If the post-treatment ranking of the
patents receiving the drug is 1,2,3,6 asses the strength of the evidence
against the relevant \(H_0\).

\paragraph{Solution}\label{solution-1}

\subsection{Exercise Three}\label{exercise-three}

\paragraph{Question}\label{question-2}

An archaeologist numbers somea rticles 1 to 11 in the order he discovers
them. He selects at random a sample of 3 of them. What is the
probability that the sum of the numbers on the items he selects is less
than or equal to 8? (You do not need to list all combinations of 3 items
from 11 to answer this question).

If the archaeologist beleived that items belonging to the more recent of
two civilizations were more likely to be found earlier in his dig and of
his 11 items 3 are identified as belong to that more recent civilization
(but the remainining 8 come from an earlier civilization) does a rank
sum of 8 for the 3 matching the more recent civilization provide
reasonable support for this theory?

\paragraph{Solution}\label{solution-2}

\subsection{Exercise Four}\label{exercise-four}

\paragraph{Question}\label{question-3}

A library has on its shelves 114 books on statistics. I take a random
sample of 12 and want to test the hypothesis that the median number of
pages, \(\theta\), in all 114 books is 225. In the sample of 12, I note
that 3 have less than 225 pages. does this justify the retention of the
hypothesis that \(\theta = 225\)? What should I take as an appropriate
alternative hypothesis? What is the largest critical region for a test
with \(P \leq 0.05\) and what is the corresponding exact \(P\)-level?

\paragraph{Solution}\label{solution-3}

\subsection{Exercise Five}\label{exercise-five}

\paragraph{Question}\label{question-4}

The numbers of pages in the sample of 12 books in Exercise 2.4 were:

\[
  126\quad142\quad156\quad228\quad245\quad246\quad370\quad419\quad433\quad454\quad478\quad503
\]

Find a condifence interval at a level not less than 95 percent for the
median \(\theta\)/

\paragraph{Solution}\label{solution-4}

\subsection{Exercise Six}\label{exercise-six}

\paragraph{Question}\label{question-5}

In Sect1.4.1 we associated a confidence interval with a two-tail test.
As well as such two-sided confidence intervals, one may define a
one-sided confidence interval composed of all parameter values that
would not be rejected in a one-tail test. Follow through such an
argument to obtain a confidence interval at level not less that 95
percent based on the sign test criteria for the 12 book sample values
given in Exercise 2.5 relevant to a test of \(H_0: \theta = \theta_0\)
against a one-sided alternative \(H_1: \theta > \theta_0\).

\paragraph{Solution}\label{solution-5}

\subsection{Exercise Seven}\label{exercise-seven}

\paragraph{Question}\label{question-6}

From 6 consenting patients requiring a medical scan, 3 are chosen at
random to undergo a positron emission tomography (PET) scan, the others
receiving a magnetic resonance imaging (MRI) scan. Image quality is
ranked in order by a hospital consultant from 1 (best) to 6 (worst).
Describe how you would test \(H_0\): \emph{scan quality is unrelated to
scan method against} (i) \(H_1:\) \emph{PET scans are better} (ii)
\(H_1:\) \emph{the scans differ in quality depending on whether they are
from PET or MRI}. Interpret the finding that the consultant rates the
three PET scans as the three highest quality images.

\paragraph{Solution}\label{solution-6}

\subsection{Exercise Eight}\label{exercise-eight}

\paragraph{Question}\label{question-7}

In Example 2.4 we remarked that a situation could arise where we might
reject \(H_0\) for the wrong reason. Explain how this is possible in
that example.

\paragraph{Solution}\label{solution-7}

\subsection{Exercise Nine}\label{exercise-nine}

\paragraph{Question}\label{question-8}

State appropriate null and alternative hypotheses for the example from
the book of Daniel about the diet in Section 1.1.3. How could you use
the ranks to calculate the probability that the four receiving the diet
of pulses were ranked 1,2,3,4? Calculate this probability assuming that
there were 20 young men involved altogether.

\paragraph{Solution}\label{solution-8}

\subsection{Exercise Ten}\label{exercise-ten}

\paragraph{Question}\label{question-9}

A sample of 12 is taken from a continuous uniform distribution over the
interval (0,1). What is the probability that the largest sample value
exceeds 0.95? (Hint: Determine the probability that any sample values
exceeds 0.95. The condition is met if at least one value exceeds 0.95.)

\paragraph{Solution}\label{solution-9}

\subsection{Exercise Eleven}\label{exercise-eleven}

\paragraph{Question}\label{question-10}

A sample of 24 is known to come either from a uniform distribution over
the interval (0,10) or else from a symmetric triangular distribution
over the same interval (0,10). The sample values are \[
4.17\quad8.42\quad3.02\quad2.89\quad9.77\quad6.06\quad2.72\quad5.12\quad6.00\quad4.78\quad2.61\quad7.20 \]
\[ 1.61\quad5.92\quad7.25\quad8.01\quad4.76\quad5.36\quad5.34\quad7.59\quad0.66\quad7.27\quad3.39\quad1.40
\]

Use appropriate graphical or other EDA techniques to get an indication
as to which of these distributions is the more likely source of the
sample.

\paragraph{Solution}\label{solution-10}


\end{document}
