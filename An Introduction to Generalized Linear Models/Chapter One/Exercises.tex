\documentclass[]{article}
\usepackage{lmodern}
\usepackage{amssymb,amsmath}
\usepackage{ifxetex,ifluatex}
\usepackage{fixltx2e} % provides \textsubscript
\ifnum 0\ifxetex 1\fi\ifluatex 1\fi=0 % if pdftex
  \usepackage[T1]{fontenc}
  \usepackage[utf8]{inputenc}
\else % if luatex or xelatex
  \ifxetex
    \usepackage{mathspec}
  \else
    \usepackage{fontspec}
  \fi
  \defaultfontfeatures{Ligatures=TeX,Scale=MatchLowercase}
\fi
% use upquote if available, for straight quotes in verbatim environments
\IfFileExists{upquote.sty}{\usepackage{upquote}}{}
% use microtype if available
\IfFileExists{microtype.sty}{%
\usepackage{microtype}
\UseMicrotypeSet[protrusion]{basicmath} % disable protrusion for tt fonts
}{}
\usepackage[margin=1in]{geometry}
\usepackage{hyperref}
\hypersetup{unicode=true,
            pdftitle={Chapter Two},
            pdfauthor={Ryan B. Honea},
            pdfborder={0 0 0},
            breaklinks=true}
\urlstyle{same}  % don't use monospace font for urls
\usepackage{longtable,booktabs}
\usepackage{graphicx,grffile}
\makeatletter
\def\maxwidth{\ifdim\Gin@nat@width>\linewidth\linewidth\else\Gin@nat@width\fi}
\def\maxheight{\ifdim\Gin@nat@height>\textheight\textheight\else\Gin@nat@height\fi}
\makeatother
% Scale images if necessary, so that they will not overflow the page
% margins by default, and it is still possible to overwrite the defaults
% using explicit options in \includegraphics[width, height, ...]{}
\setkeys{Gin}{width=\maxwidth,height=\maxheight,keepaspectratio}
\IfFileExists{parskip.sty}{%
\usepackage{parskip}
}{% else
\setlength{\parindent}{0pt}
\setlength{\parskip}{6pt plus 2pt minus 1pt}
}
\setlength{\emergencystretch}{3em}  % prevent overfull lines
\providecommand{\tightlist}{%
  \setlength{\itemsep}{0pt}\setlength{\parskip}{0pt}}
\setcounter{secnumdepth}{0}
% Redefines (sub)paragraphs to behave more like sections
\ifx\paragraph\undefined\else
\let\oldparagraph\paragraph
\renewcommand{\paragraph}[1]{\oldparagraph{#1}\mbox{}}
\fi
\ifx\subparagraph\undefined\else
\let\oldsubparagraph\subparagraph
\renewcommand{\subparagraph}[1]{\oldsubparagraph{#1}\mbox{}}
\fi

%%% Use protect on footnotes to avoid problems with footnotes in titles
\let\rmarkdownfootnote\footnote%
\def\footnote{\protect\rmarkdownfootnote}

%%% Change title format to be more compact
\usepackage{titling}

% Create subtitle command for use in maketitle
\newcommand{\subtitle}[1]{
  \posttitle{
    \begin{center}\large#1\end{center}
    }
}

\setlength{\droptitle}{-2em}
  \title{Chapter Two}
  \pretitle{\vspace{\droptitle}\centering\huge}
  \posttitle{\par}
  \author{Ryan B. Honea}
  \preauthor{\centering\large\emph}
  \postauthor{\par}
  \date{}
  \predate{}\postdate{}


\begin{document}
\maketitle

\subsection{Exercise One}\label{exercise-one}

\paragraph{Question}\label{question}

Genetically similar seeds are randomly assigned to be raised in either a
nutritionally entriched environment (treatment group) or standard
conditions (control group) usig a completely randomized experimental
design. After a predetermined time, all plants are harvested, dried and
weighed. The results, expressed in grams, for 20 plants in each group
are shown in the table below.

\begin{center}
\begin{tabular}{@{}cccc@{}}
\toprule
\multicolumn{2}{l}{Treatment Group} & \multicolumn{2}{l}{Control Group} \\ \midrule
4.81             & 5.36             & 4.17            & 4.66            \\
4.17             & 3.48             & 3.05            & 5.58            \\
4.41             & 4.69             & 5.18            & 3.66            \\
3.59             & 4.44             & 4.01            & 4.50            \\
5.87             & 4.89             & 6.11            & 3.90            \\
3.83             & 4.71             & 4.10            & 4.61            \\
6.03             & 5.48             & 5.17            & 5.62            \\
4.98             & 4.32             & 3.57            & 4.53            \\
4.90             & 5.15             & 5.33            & 6.05            \\
5.75             & 6.34             & 5.59            & 5.14            \\ \bottomrule
\end{tabular}
\end{center}

We want to test whether there is any difference in yield between the two
groups. Let \(Y_{jk}\) denote the \(k\)th observation in the \(j\)th
group where \(j = 1\) for the treatment group, \(j = 2\) for the control
group and \(k = 1,...,20\) for both groups. Assume that the \(Y_{jk}\)'s
are independent random variables with
\(Y_{jk} \sim \text{N}(\mu_j, \sigma^2)\). The null hypothesis
\(H_0: \mu_1 = \mu_2 = \mu\), that there is no differenc, is to be
compared with the alternative hypothesis \(H_1 : \mu_1 \neq \mu_2\).

\textbf{(a):} Conduct a exploratory analysis of the data looking at the
distributions for each group (e.g.~using dot plots, stem and leaf plots
or Normal probability plots) and calculate summary statistics (e.g.,
means, medians, standard derivations, maxima and minima). What can you
infer from these investigations?

\textbf{(b):} Perform an unpaired t-test on these data and calculate a
95\% confidence interval for the difference between the group means.
Interpret these results.

\textbf{(c):} The following models can be used to test the null
hypothesis \(H_0\) against the alternative hypothesis \(H_1\), where

\begin{align*}
H_0 : E(Y_{jk}) &= \mu;\quad Y_{jk} \sim \text{N}(\mu, \sigma^2),\\
H_1 : E(Y_{jk}) &= \mu_j;\quad Y_{jk} \sim \text{N}(\mu_j, \sigma^2),
\end{align*}

for \(j = 1,2\) and \(k = 1,...,20\). Find the maximum likelihood and
least squares estimate of the parameters \(\mu, \mu_1, \mu_2\) assuming
\(\sigma^2\) is a known constant.

\textbf{(d):} Show that the minimum values of the least squares criteria
area

\begin{align*}
\text{for } H_0, \quad \hat{S_0} &=  \sum \sum (Y_{jk} - \overline{Y})^2, \text{ where } \overline{Y} = \sum_{j=1}^2\sum_{k=1}^K Y_{jk}/40;\\
\text{for } H_1, \quad \hat{S_1} &= \sum \sum (Y_{jk} - \overline{Y_j})^2, \text{ where } \sum_{k=1}^K Y_{jk}/40;
\end{align*}

for \(j = 1,2\).

\textbf{(e):} Using the results of (d) show that \[
\frac{1}{\sigma^2}\hat{S_1} = \frac{1}{\sigma^2}\sum_{j=1}^2\sum_{k=1}^{20}(Y_{jk} - \mu_j)^2 - \frac{20}{\sigma^2}\sum^{20}_{k=1}(\overline{Y_j} - \mu_j)^2,
\] and deduce that if \(H_1\) is true \[
\frac{1}{\sigma^2}\hat{S_1} \sim \chi^2(38).
\] Similarly show that \[
\frac{1}{\sigma^2}\hat{S_0} = \frac{1}{\sigma^2}\sum_{j=1}^2\sum_{k=1}^{20}(Y_{jk} - \mu)^2 - \frac{40}{\sigma^2}\sum_{j=1}^2(\overline{Y} - \mu)^2
\] and if \(H_0\) is true then \[
\frac{1}{\sigma^2}\hat{S_0} \sim \chi^2(39).
\]

\textbf{(f):} Use an argument similar to the one in Example 2.2.2 and
the results from (e) to deduce that the statistic \[ 
F = \frac{\hat{S_0} - \hat{S_1}}{\hat{S_1}/38}
\] has the central \(F\)-distribution \(F(1,38)\) if \(H_0\) is true and
a non-central distribution if \(H_0\) is not true.

\textbf{(g):} Calculate the \(F\)-statistic from (f) and use it to test
\(H_0\) against \(H_1\). What do you conclude?

\textbf{(h):} Compare the value of \(F\)-statistic from (g) with the
t-statistic from (b), recalling the relationship between the
\(t\)-distribution and the \(F\)-distribution (see Section 1.4.4). Also
compare the conclusion from (b) and (g).

\textbf{(i):} Calculate residuals from the model for \(H_0\) and use
them to explore the distributional assumptions.

\paragraph{Solution}\label{solution}

\textbf{(a):}

\textbf{(b):}

\textbf{(c):}

\textbf{(d):}

\textbf{(e):}

\textbf{(f):}

\textbf{(g):}

\textbf{(h):}

\textbf{(i):}

\subsection{Exercise Two}\label{exercise-two}

\paragraph{Question}\label{question-1}

The weights, in kilograms, of twenty men before and after participation
in a ``waist loss'' program are shown in Table 2.8 (Egger et al. 1999).
We want to know if, on average, they retain a weight loss twelve months
after the program.

\begin{center}
\begin{tabular}{@{}ccccccc@{}}
\toprule
Man & Before & After &  & Man & Before & After \\ \cmidrule(r){1-3} \cmidrule(l){5-7} 
1   & 100.8  & 97    &  & 11  & 105    & 105   \\
2   & 102    & 107.5 &  & 12  & 85     & 82.4  \\
3   & 105.9  & 97    &  & 13  & 107.2  & 98.2  \\
4   & 108    & 108   &  & 14  & 80     & 83.6  \\
5   & 92     & 84    &  & 15  & 115.1  & 115   \\
6   & 116.7  & 111.5 &  & 16  & 103.5  & 103   \\
7   & 110.2  & 102.5 &  & 17  & 82     & 80    \\
8   & 135    & 127.5 &  & 18  & 101.5  & 101.5 \\
9   & 123.5  & 118.5 &  & 19  & 103.5  & 102.6 \\
10  & 95     & 94.2  &  & 20  & 93     & 93    \\ \bottomrule
\end{tabular}
\end{center}

Let \(Y_jk\) denote the weight of the \(k\)th man at the \(j\)th time,
where \(j = 1\) before the program and \(j = 2\) twelve months later.
Assume the \(Y_{jk}\)'s are independent random variables
\(Y_{jk} \sim N(\mu_j, \sigma^2)\) for \(j = 1,2\) and \(k = 1,...,20\).

\textbf{(a):} Use an unpaired t-test to test the hypothesis \[
H_0 : \mu_1 = \mu_2 \quad \text{versus} \quad H_1: \mu_1 \neq \mu_2.
\]

\textbf{(b):} Let \(D_k = Y_{1k} - Y_{2k}\), for \(k = 1,...,20\).
Formulate models for testing \(H_0\) against \(H_1\) using the
\(D_k\)'s. Using analogous methods to Exercise 2.1 above, assuming
\(\sigma^2\) is a known constant, test \(H_0\) against \(H_1\).

\textbf{(c):} The analysis in (b) is a paired t-test which uses the
natural relationship between weights of the \emph{same} person before
and after the program. Are the conclusions the same from (a) and (b)?

\textbf{(d):} List the assumptions made for (a) and (b). Which analysis
is more appropriate for these data?

\paragraph{Solution}\label{solution-1}

\textbf{(a):}

\textbf{(b):}

\textbf{(c):}

\textbf{(d):}

\subsection{Exercise Three}\label{exercise-three}

\paragraph{Question}\label{question-2}

For model (2.7) for the data on birthweight and gestational age, using
methods similar to those for Exercise 1.4, show

\begin{align*}
\hat{S_1} &= \sum_{j=1}^J\sum_{k=1}^K(Y_{jk} - a_j - b_jx_{jk})^2\\
 &= \sum_{j=1}^J\sum_{k=1}^K[(Y_{jk} - (a_j + \beta_jx_{jk})]^2 - K\sum_{j=1}^J(\overline{Y_j} - \alpha_j - \beta_j\overline{x_j})^2\\
 &-\sum_{j=1}^J(b_j - \beta_j)^2(\sum_{k=1}^K x_{jk}^2 - K\overline{x_j^2})
\end{align*}

and that the random variables \(Y_{jk}\), \(\overline{Y_j}\) and \(b_j\)
are all independent and have the following distributions

\begin{align*}
Y_{jk} &\sim \text{N}(\alpha_j + \beta_jx_{jk}, \sigma^2),   \\
\overline{Y_j} &\sim \text{N}(\alpha_j + \beta_j\overline{x_j}, \sigma^2/K),    \\
b_j &\sim \text{N}(\beta_j, \sigma^2/(\sum_{k=1}^K x_{jk}^2 - K\overline{x^2_j})).
\end{align*}

\paragraph{Solution}\label{solution-2}

\subsection{Exercise Four}\label{exercise-four}

Suppose you have the following data:

\begin{center}
\begin{tabular}{lrrrrrr}
\hline
x: & 1.0  & 1.2  & 1.4  & 1.6  & 1.8  & 2.0   \\
y: & 3.15 & 4.85 & 6.50 & 7.20 & 8.25 & 16.50 \\ \hline
\end{tabular}
\end{center}

and you want to fit a model with \[
\text{E}(Y) = \ln(\beta_0 + \beta_1x + \beta_2x^2).
\]

\paragraph{Solution}\label{solution-3}

\subsection{Exercise Five}\label{exercise-five}

THe model for two-factor analysis of variance with two levels of one
factor, three levels of the other ad no replication is \[
\text{E}(Y_{jk}) = \mu_{jk} = \mu + \alpha_j + \beta_{k}; \quad \quad Y_{jk} \sim \text{N}(\mu_{jk}, \sigma^2),
\] where \(j = 1,2; k = 1,2,3\) and, using the sum-to-zero constraints,
\(\alpha_1 + \alpha_2 = 0, \beta_1 + \beta_2 = 0\). Also the
\$Y\_\{jk\}'s are assumed to be independent.

Write the equation for \(\text{E}(Y_{jk})\) in matrix notation. (Hint:
Let \(\alpha_2 = -\alpha_1\) and \(\beta_3 = -\beta_1 - \beta_2\))

\paragraph{Solution}\label{solution-4}


\end{document}
