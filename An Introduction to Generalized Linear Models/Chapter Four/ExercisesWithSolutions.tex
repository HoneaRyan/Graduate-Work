\documentclass[]{article}
\usepackage{lmodern}
\usepackage{amssymb,amsmath}
\usepackage{ifxetex,ifluatex}
\usepackage{fixltx2e} % provides \textsubscript
\ifnum 0\ifxetex 1\fi\ifluatex 1\fi=0 % if pdftex
  \usepackage[T1]{fontenc}
  \usepackage[utf8]{inputenc}
\else % if luatex or xelatex
  \ifxetex
    \usepackage{mathspec}
  \else
    \usepackage{fontspec}
  \fi
  \defaultfontfeatures{Ligatures=TeX,Scale=MatchLowercase}
\fi
% use upquote if available, for straight quotes in verbatim environments
\IfFileExists{upquote.sty}{\usepackage{upquote}}{}
% use microtype if available
\IfFileExists{microtype.sty}{%
\usepackage{microtype}
\UseMicrotypeSet[protrusion]{basicmath} % disable protrusion for tt fonts
}{}
\usepackage[margin=1in]{geometry}
\usepackage{hyperref}
\hypersetup{unicode=true,
            pdftitle={Chapter Four},
            pdfauthor={Ryan Honea},
            pdfborder={0 0 0},
            breaklinks=true}
\urlstyle{same}  % don't use monospace font for urls
\usepackage{longtable,booktabs}
\usepackage{graphicx,grffile}
\makeatletter
\def\maxwidth{\ifdim\Gin@nat@width>\linewidth\linewidth\else\Gin@nat@width\fi}
\def\maxheight{\ifdim\Gin@nat@height>\textheight\textheight\else\Gin@nat@height\fi}
\makeatother
% Scale images if necessary, so that they will not overflow the page
% margins by default, and it is still possible to overwrite the defaults
% using explicit options in \includegraphics[width, height, ...]{}
\setkeys{Gin}{width=\maxwidth,height=\maxheight,keepaspectratio}
\IfFileExists{parskip.sty}{%
\usepackage{parskip}
}{% else
\setlength{\parindent}{0pt}
\setlength{\parskip}{6pt plus 2pt minus 1pt}
}
\setlength{\emergencystretch}{3em}  % prevent overfull lines
\providecommand{\tightlist}{%
  \setlength{\itemsep}{0pt}\setlength{\parskip}{0pt}}
\setcounter{secnumdepth}{0}
% Redefines (sub)paragraphs to behave more like sections
\ifx\paragraph\undefined\else
\let\oldparagraph\paragraph
\renewcommand{\paragraph}[1]{\oldparagraph{#1}\mbox{}}
\fi
\ifx\subparagraph\undefined\else
\let\oldsubparagraph\subparagraph
\renewcommand{\subparagraph}[1]{\oldsubparagraph{#1}\mbox{}}
\fi

%%% Use protect on footnotes to avoid problems with footnotes in titles
\let\rmarkdownfootnote\footnote%
\def\footnote{\protect\rmarkdownfootnote}

%%% Change title format to be more compact
\usepackage{titling}

% Create subtitle command for use in maketitle
\newcommand{\subtitle}[1]{
  \posttitle{
    \begin{center}\large#1\end{center}
    }
}

\setlength{\droptitle}{-2em}
  \title{Chapter Four}
  \pretitle{\vspace{\droptitle}\centering\huge}
  \posttitle{\par}
  \author{Ryan Honea}
  \preauthor{\centering\large\emph}
  \postauthor{\par}
  \date{}
  \predate{}\postdate{}

\usepackage{bbm}
\usepackage{bm}

\begin{document}
\maketitle

\subsection{Exercise One}\label{exercise-one}

\paragraph{Question}\label{question}

The data in Table the table below show the numbers of cases of AIDS in
Australia by date of diagnosis for successive 3-months periods from 1984
to 1988. (Data from National Centre for HIV Epidemiology and Clinical
Research 1994.)

In this early phase of the epidemic, the numbers of cases seemed to be
increasing exponentially.

\begin{center}
\begin{tabular}{@{}crrrr@{}}
\toprule
     & \multicolumn{4}{c}{Quarter} \\
Year & 1     & 2     & 3    & 4    \\ \midrule
1984 & 1     & 6     & 16   & 23   \\
1985 & 27    & 39    & 31   & 30   \\
1986 & 43    & 51    & 63   & 70   \\
1987 & 88    & 97    & 91   & 104  \\
1988 & 110   & 113   & 149  & 159  \\ \bottomrule
\end{tabular}
\end{center}

\paragraph{Solutions}\label{solutions}

\textbf{(a)}: Plot the number of cases \(y_i\) against time period \(i\)
(\(i = 1,...,20\)).

\textbf{(b)}: A possible model is the Poisson distribution with
parameter \(\lambda_i = i^\theta\), or equivelantly \[
\log\lambda_i = \theta\log i.
\] Plot \(\log y_i\) against \(\log i\) to examine this model.

\textbf{(c)}: Fit a gneralized linear model to these data using the
Poisson distribution, the log-link function and the equation \[
g(\lambda_i) = \log\lambda_i = \beta_1 + \beta_2x_i,
\] where \(x_i = \log i\). Firstly, do this from first principles,
working out expressions for the weight matrix \(\bm{W}\) and other terms
needed for the iterative equation \[
\bm{X}^T\bm{WXb}^{(m)} = \bm{X}^T\bm{Wz}
\] and using software which can perform matrix operations to carry out
the calculations.

\textbf{(d)}: Fit the model in (c) using statistical software which can
perform Poisson regression. Compare the results with those obtained in
(c).

\pagebreak

\subsection{Exercise Two}\label{exercise-two}

\paragraph{Question}\label{question-1}

The data in the table below are times to death, \(y_i\), in weeks from
diagnosis and \(\log_{10}\)(initial white blood cell count), \(x_i\),
for seventeen patients suffering from leukemia. (This is Example U from
Cox and Snell 1981.)

\begin{center}
% Please add the following required packages to your document preamble:
% \usepackage{booktabs}
\begin{table}[]
\centering
\caption{My caption}
\label{my-label}
\begin{tabular}{@{}crrrrrrrrr@{}}
\toprule
$y_i$ & 65   & 156  & 100  & 134  & 16   & 108  & 121  & 4    & 39   \\
$x_i$ & 3.36 & 2.88 & 3.63 & 3.41 & 3.78 & 4.02 & 4.00 & 4.23 & 3.73 \\
      &      &      &      &      &      &      &      &      &      \\
$y_i$ & 143  & 56   & 26   & 22   & 1    & 1    & 5    & 65   &      \\
$x_i$ & 3.85 & 3.97 & 4.51 & 4.54 & 5.00 & 5.00 & 4.72 & 5.00 &      \\ \bottomrule
\end{tabular}
\end{table}
\end{center}

\paragraph{Solutions}\label{solutions-1}

\textbf{(a)}: Plot \(y_i\) against \(x_i\). Do the data show any trends?

\textbf{(b)}: A possible specification for \(E(Y)\) is \[
\text{E}(Y_i) = \exp(\beta_1 + \beta_2x_i),
\] which can ensure all E\((Y)\) is non-negative for lall values of the
paramaters and all values of \(x\). Which link functio is appropriate in
this case?

\textbf{(c)}: The Exponential distribution is ofte used to describe
survival times. The probability distribution is
\(f(y; \theta) = \theta e^{-y\theta}\). This is a sepcial case of the
Gamma distribution with shape parameters \(\phi = 1\) (see Exercise
3.12(a)). Show that E\((Y) = 1/\theta\) and var\((Y) = 1/\theta^2\).

\textbf{(d)}: Fit a model with the equation for E\((Y_i)\) given in (b)
and the Exponential distribution using appropriate statistical software.

\textbf{(e)}: For the model fitted in (d), compare the observed values
\(y_i\), and fitted values
\(\hat{y_i} = \exp(\hat{\beta_1} + \hat{\beta_2}x_i),\) and use the
standardized residuals \(r_i = (y_i - \hat{y_i})/\hat{y_i}\) to
investigate the adequacy of the model. (Note: \$hat\{y\_i\}\$ is used as
the denominator of \(r_i\) because it is an estimate of the standard
deviation of \(Y_i\)--see (c) above.)


\end{document}
